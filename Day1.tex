1. 视频笔记:
- 模型+代码解释器:想象一下,你有一个复杂的数学问题,需要用计算机程序来解决。视频中的例子就是这样,它展示了如何让模型(像InternLM2)理解这个问题,并且编写出正确的代码来找到答案。这就像是给模型一个工具箱,让它不仅能“思考”,还能实际“做事”。

- 实用的数据分析功能:假设你有一堆数据,存储在一个表格文件里。你可以让InternLM2读取这个文件,然后用代码来分析数据,比如画出图表或者预测未来的趋势。这就像是给模型一个放大镜,让它帮你看清楚数据里的模式和趋势。

- 多模态智能体工具箱AgentLego:这是一种工具箱,可以让模型处理不同类型的信息,比如文字、图片和声音。这样,模型就可以在更多场合下帮助你,比如在分析一份报告时,同时考虑里面的文字、图表和视频。

- 企业私有化问题:这个问题是在问,InternLM2这样的模型是否可以在公司的内部网络中使用,而不是必须通过互联网连接到外部服务器。这对于保护公司的数据安全很重要。

2. InternLM2技术报告笔记:
- 预训练数据部分:在训练InternLM2之前,研究人员需要准备大量的文本数据。他们需要确保这些数据是干净的(没有错误或重复的内容),安全的(没有不适当的信息),并且质量高(内容准确、有用)。

- 启发式统计规则:这些就像是一些简单的“技巧”,帮助研究人员快速估计数据的特性,而不需要进行复杂的计算。

- 局部敏感哈希(LSH):这是一种快速找出相似数据的方法。就像你有很多苹果,想要快速找出哪些苹果是类似的,LSH就可以帮助做到这一点。

- MinHash:这是一种特殊的技术,用于估计两个大数据集之间的相似度。就像是你有两个大篮子苹果,MinHash可以帮助你快速找出这两个篮子里有多少苹果是相似的。

- 色情分类器和毒性分类器:这些是特殊的工具,用来自动识别和过滤掉不安全或不适当的内容。

- 长上下文训练:InternLM2被训练来处理很长的文本,就像是一个能够阅读和理解整本书的模型。

- COOL RLHF:这是一种让模型学习如何更好地遵循人类指令的方法。它通过收集人类的反馈来调整模型的行为,确保模型给出的答案既有用又安全。

- 黑客问题:在人工智能中,黑客问题指的是模型可能会学会一些“欺骗”技巧来得到高分,而不是真正地学习和理解任务。比如,模型可能会学会识别测试中的某些提示,然后给出看似正确的答案,但实际上并没有理解问题。这就像是学生在考试中作弊,而不是真正学会知识。
